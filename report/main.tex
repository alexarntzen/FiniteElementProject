\documentclass[5pt,a4paper,english]{elsarticle}% 5p gir 2 kolonner pr side. 1p gir 1 kolonne pr side.
%\documentclass[11pt,a4paper,norsk]{article} % setter hvilken type dokument. Kan også være book eller report. I klammeparantes settes fontstørrelse, papirstørrelse og språk.

\usepackage[utf8]{inputenc} %- Løser problem med å skrive andre enn engelske bokstaver f.eks æ,ø,å.

\usepackage[T1]{fontenc} %- Støtter koding av forskjellige fonter.
\usepackage{lmodern}
\usepackage{textcomp} % Støtter bruk av forskjellige fonter som dollartegn, copyright, en kvart, en halv mm, se http://gcp.fcaglp.unlp.edu.ar/_media/integrantes:psantamaria:latex:textcomp.pdf

\usepackage{url} % Gjør internett- og e-mail adresser klikkbare i tex-dokumentet.

\usepackage{hyperref} % Gjør referansene i tex-dokumentet klikkbare, slik at du kommer til referansen i referanselista.

\usepackage[english]{babel} % Ordbok. Hvis man setter norsk i options til usepackage babel kan man bruke norske ord.

\usepackage{natbib}
\bibliographystyle{unsrtnat}

\urlstyle{sf} % Velger hvilken stil url-adresser skrives, f.eks sf

\usepackage{graphicx} % Brukes for å sette inn bilder eller figurer
\usepackage{amsmath} 				% Ekstra matematikkfunksjoner.
\usepackage{amssymb}
\usepackage{amsfonts}
\usepackage{amsthm}
\usepackage{mathrsfs}
\usepackage{mathtools}
\usepackage{geometry}
\usepackage{tikz-cd}
\usepackage{bm}
\usepackage{graphicx}
\usepackage{changepage}
\usepackage{subcaption}
\usepackage{placeins}
\usepackage{bm}
\usepackage{physics}



\usepackage{siunitx}					% Må inkluderes for blant annet å få tilgang til kommandoen \SI (korrekte måltall med enheter)
	\sisetup{exponent-product = \cdot}      	% Prikk som multiplikasjonstegn (i steden for kryss).
 	\sisetup{output-decimal-marker  =  {,}} 	% Komma som desimalskilletegn (i steden for punktum).
 	\sisetup{separate-uncertainty = true}   	% Pluss-minus-form på usikkerhet (i steden for parentes). 

\usepackage{booktabs} % For å få tilgang til finere linjer (til bruk i tabeller og slikt).

\usepackage[font=small,labelfont=bf]{caption}		% For justering av figurtekst og tabelltekst.


\journal{ }
\usepackage{etoolbox}
\makeatletter
\patchcmd{\ps@pprintTitle}
  {Preprint submitted to}
  {}
  {}{}
\makeatother
% Fjerner submitte dto 

% math stuff
\newcommand{\restr}[2]{\ensuremath{\left.#1\right|_{#2}}}

% my personal commands
\newcommand{\R}{\mathbb{R}}

% Denne setter navnet på abstract til Sammendrag
%\renewenvironment{abstract}{\global\setbox\absbox=\vbox\bgroup
%\hsize=\textwidth\def\baselinestretch{1}%
%\noindent\unskip\textbf{Sammendrag}
%\par\medskip\noindent\unskip\ignorespaces}
%{\egroup}

%\clearpage % Bruk denne kommandoen dersom du vil ha ny side etter det er satt plass til figuren.
% Disse kommandoene kan gjøre det enklere for LaTeX å plassere figurer og tabeller der du ønsker.
\setcounter{totalnumber}{5}
\renewcommand{\textfraction}{0.05}
\renewcommand{\topfraction}{0.95}
\renewcommand{\bottomfraction}{0.95}
\renewcommand{\floatpagefraction}{0.35}


% math stuff
\newtheorem{theorem}{Theorem}
\newtheorem{claim}[theorem]{Claim}
\newtheorem{proposition}[theorem]{Proposition}
\newtheorem{lemma}[theorem]{Lemma}
\newtheorem{corollary}[theorem]{Corollary}
\newtheorem{conjecture}[theorem]{Conjecture}
\newtheorem*{observation}{Observation}
\newtheorem*{example}{Example}
\newtheorem*{remark}{Remark}

\graphicspath{{../}}

%%%%%%%%%%%%%%%%%%%%%%%%%%%%%%%%%%%%%%%%%%%%%%%%%%%%%%%%%%%%%%%%%%%%%%%%%
\begin{document}

\begin{frontmatter}

\title{Poisson and elasticity problems solved with finite elements}
\author[matematikk]{Håkon Noren}

\author[matematikk]{Alexander Johan Arntzen }

\address[matematikk]{Department of Mathematical Science, Norwegian University of Science and Technology, N-7491 Trondheim, Norway.}


\begin{abstract}
This report describes the implementation and performance of a finite element method used to solve differential equations in two dimensions. The Poisson problem is solved, first with homogeneous Dirichlet boundary conditions, secondly with mixed Neumann and Dirichlet boundary conditions. Furthermore numerical solutions of the elasticity problem, modelling displacement and stress, are described.

The solutions are compared to the analytical solutions and for the elasticity problem the numerical convergence rate is reported.
\end{abstract}

\end{frontmatter}


%%%%%%%%%%%%%%%%%%%%%%%%%%%%%%%%%%%%%%%%%%%%%%%%%%%%%%%%%%%%%%%%%%%%%%%%%
\section{Introduction}
The theory section describes two differential equations; the Poission problem, and the elasticity problem. The corresponding weak formulation is derived for both problems and projected into finite subspaces with the Galerkin approach, for the implementation of a finite element scheme. In order to test the schemes, two different test equations with known analytical solutions are described.

Finally the numerical solutions are plotted before numerical convergence and error is discussed.

%%%%%%%%%%%%%%%%%%%%%%%%%%%%%%%%%%%%%%%%%%%%%%%%%%%%%%%%%%%%%%%%%%%%%%%%%
\section{The Poisson problem}

\subsection{Boundary value problem}

We consider the Poisson problem with mixed homogenous Dirichlet and Neumann boundary conditions. 

\begin{equation}
\begin{aligned}
    \nabla^2 u &= - f \\
    \restr{u}{\partial \Omega_{D}}  &= 0 \\
    \restr{\frac{\partial u}{\partial n}}{\partial \Omega_{N}} &= g,
\label{poission-problem}
\end{aligned}
\end{equation}
on the domain $\Omega$ given by the unit disk

\begin{equation*}
    \Omega = \{(x,y) : x^2+ y^2 \leq 1 \}.
\end{equation*}
\begin{remark}
Note that for $\restr{u}{\partial \Omega_D} = 0$, when $\partial \Omega_{D} = \partial \Omega$ we have the homogenous Dirichlet boundary conditions. Also, when $\partial \Omega_D = \emptyset$ the poission problem does not have a unique solution. In the rest of the text we assume $\partial \Omega_D \neq \emptyset$ 
\end{remark}



\subsection{Analytical solution}
We consider an analytical solution, $u$, to the Poission problem \eqref{poission-problem}
\begin{proposition}\label{prop-analytical-solution}
    $u: \mathbb{R}^2 \rightarrow \mathbb{R}$ such that

    \begin{equation}
    \restr{u(x,y)}{\Omega} = \sin(2\pi (x^2 + y^2)),
    \label{analytical-solution}
    \end{equation}
    is a solution to the poission problem \eqref{poission-problem} with homogenous Dirichlet boundary conditions, and  $f$ in polar coordinates given by

    \begin{equation}
    f(r,\theta) = -8\pi\cos(2\pi r^2) + 16 \pi^2r^2 \sin(2\pi r^2)
    \label{f-analytic}
    \end{equation}
\end{proposition}
\begin{proof}
By calculation

\begin{equation*}
\begin{aligned}
 u_{xx} &= 4\pi\cos(2\pi r^2) - 16\pi^2x^2\sin(2\pi r^2)
\\
u_{yy} &= 4\pi\cos(2\pi r^2) - 16\pi^2y^2\sin(2\pi r^2),
\end{aligned}
\end{equation*}
As $r^2 = x^2 + y^2$ we find by comparison
\begin{equation*}
\begin{aligned}
\nabla^2 u &= u_{xx} + u_{yy}\\
&= 8\pi \cos(2\pi r^2) - 16\pi^2r^2\sin(2\pi r^2)\\
& = -f(x,y).
\end{aligned}
\end{equation*}
The boundary conditions are then checked
\begin{equation*}
    \restr{u(x,y)}{\partial\Omega} = \sin(2\pi) = 0,
\end{equation*}
and we find that $u$ fulfills all the conditions given in \eqref{poission-problem}. 
\end{proof}
\noindent Then $u$ also hold for the corresponding solution with Neumann bundary conditions. 

\begin{proposition}\label{analytical-solution-neumann}
$u$ defined in Equation \eqref{analytical-solution} satisfies Equation \eqref{poission-problem} with $g$ given in polar coordinates as
\begin{equation}
    \restr{g(r,\theta)}{\Omega_N} =  4\pi r\cos(2\pi r^2).
    \label{g-analytic}
\end{equation}
\end{proposition}
\begin{proof}
    The surface normal $\boldsymbol n$ to the unit disk $\Omega$ is the unit vector in the radial direction. Then 
    \begin{equation*}
        \frac{\partial u}{\partial n} = \frac{\partial u}{\partial r},
    \end{equation*}
    and by calculation in polar coordinates
    \begin{equation*}
        \frac{\partial u}{\partial r} = 4 \pi r \sin(2\pi r^2) = g(r,\theta).
    \end{equation*}
\end{proof}

\subsection{Weak formulation}

\begin{proposition}
    The weak formulation of \eqref{poission-problem} with homogeneous dirichlet boundary conditions is given by:  
    \begin{equation*}
        \begin{aligned}
        &\text{find } u \in H_0^{1}(\Omega) : 
        \\
        &a(u,v) = l(v) \ \forall v \in H_0^{1}(\Omega), 
        \end{aligned}
    \end{equation*}
    where $a$ is the bilinear functional and $l$ the linear functional defined as 
    \begin{equation*}
        \begin{aligned}
            a(u,v) &= \iint\limits_{\Omega} \nabla u \cdot \nabla v \, dx \, dy
            \\
            l(v) &= \iint\limits_{\Omega} fv \, dx \, dy.
        \end{aligned}
    \end{equation*}
\end{proposition}
\begin{proof}
    This proof is heavily influenced by Quarteroni\cite[p. 44-45]{AQuart}.
    Assume $u$ satisfies \eqref{poission-problem}. Then mulitplying by a test function $v$ and integration over $\Omega$ results in the following integral equation
    \begin{equation}
        \label{weak-start}
        \begin{aligned}
            \iint\limits_{\Omega} \nabla^2uv \, dx \, dy &=  -\iint\limits_{\Omega} fv \, dx \, dy.\ 
        \end{aligned}
    \end{equation}
    Applying Green's first identity and the fact that $\frac{\partial u}{\partial n} = 0$ on $\partial \Omega$ gives
    \begin{equation}
        \iint\limits_{\Omega} \nabla^2 u v \, dx \, dy 
        = \underbrace{\int_{\partial \Omega} \frac{\partial u}{\partial n} v \, d\gamma}_{=0} - \iint\limits_{\Omega} \nabla u \cdot \nabla v  \, dx \, dy,
        \label{greens-identety}
    \end{equation}
    which we can apply to get 
    \begin{equation}
        \iint\limits_{\Omega} \nabla u \cdot \nabla v \, dx \, dy = 
        \iint\limits_{\Omega} fv \, dx \, dy.
    \end{equation}
    
    Since $v$ was arbitrary the weak formulation follows. Furthermore, all $u$ and $v$ must belong to a space such that $\nabla u \cdot \nabla v \in {L^1}(\Omega)$, where the derivatives are of the weak form. Then $\nabla u$ and $\nabla v$ must be in $\left[{L^2}(\Omega)\right]^2$, while $u$ and $v$ must be in ${H^1_0}(\Omega)$. Finally, linearity and bilinearity of $l$ and $a$ follows directly from linearity of the weak derivative and the integral.
\end{proof}
\begin{remark}
    When Neumann boundary conditions are considered we have to change the space of test functions to 
    \begin{equation}
        H^1_{\partial \Omega_D } = \{v \in H^1(\Omega):\restr{v}{\partial \Omega_D }=0 \}.
    \end{equation}
    This changes $a$ and $l$, as they now operate on a new function space. In addition the expression for $l$ changes to
    \begin{equation}
        l(v) = \int_{\partial \Omega_N} \underbrace{ \frac{\partial u}{\partial n}}_{=g} v \, d\gamma + \iint\limits_{\Omega} f v  \, dx \, dy.
        \label{neumann-conditions}
    \end{equation}
\end{remark}

\subsection{Galerkin projection}

We now search for solutions in the finite subspace $X_h \subset X$. 
By discretizing $\Omega$ into $M$ triangles $K_k$, as seen in Figure \ref{plot-meshes}, defined by corner nodes $(x_i,y_i)$, we have basis functions corresponding to each node.
Let 
\begin{equation*}
X_h = \{ v \in X : v|_{K_k} \in \mathbb{P}_1 (K_k),1\leq k\leq M \},
\end{equation*}
and $\{\phi_i\}_{i=1}^n$ be the basis functions of $X_h$ such that

\begin{equation*}
    \begin{aligned}
X_h = \text{span} \{\phi_i\}_{i=1}^n & & \phi_j(x_i,y_i) = \delta_{ij}.
    \end{aligned}
\end{equation*}
Hence we aim to find $u_h \in X_h$ $\forall v \in X_h$, meaning we can write
\begin{equation*}
    \begin{aligned}
u_h = \sum_{i=1}^n u_i \phi_i(x,y) & & v_h = \sum_{j=1}^n v_j \phi_j(x,y).
    \end{aligned}
\end{equation*}
\noindent with $u_i,v_j \in \mathbb{R}$. Hence we get the weak formulation
\begin{equation}
\begin{aligned}
a(u_h,v_h) &= l(v_h) 
\\
\iint\limits_{\Omega} 
\sum_{i=1}^n u_i \nabla \phi_i \sum_{j=1}^n v_j \nabla\phi_j \, dx \, dy 
&= \iint\limits_{\Omega} f \sum_{j=1}^n v_j \phi_j \, dx \, dy
\\
\sum_{i=1}^n\sum_{j=1}^n u_i v_j a(\phi_i,\phi_j) &= \sum_{j=1}^n v_j l(\phi_i)
\\
\bm v^T\bm A \bm u &= \bm v^T \bm f \quad \forall \bm v \in \mathbb R^n
\\ \bm A \bm u &= \bm f.
\label{weak_formulation_matrix}
\end{aligned}
\end{equation}
where we use the linearity of $a,l$. 
The aim is now to find $u_h$ satisfying equation \eqref{weak_formulation_matrix}, $\forall \bm v \in \mathbb R^n$.
With $\bm A, \bm u, \bm f$ given by

\begin{equation}
    \begin{aligned}
\bm A &= [a(\phi_i,\phi_j)]_{i,j}
\\ 
\bm u &= [u_i]_i
\\
\bm f &= [l(\phi_j)]_j & i,j = 1,\cdots,n
\end{aligned}
\end{equation}
\begin{remark}
    In the case for homogeneous Dirichlet boundary conditions $X = H_0^1(\Omega)$, with $X_h$ a subspace of $H_0^1(\Omega)$ 
\end{remark}
\begin{remark}
Note that in equation \eqref{weak_formulation_matrix} $\bm A$ is singular. 
This is a consequence of the fact that no boundary conditions has yet been enforced on the systems of equations.
The Poisson problem stated in \eqref{poission-problem} will have infinitely many solutions $u$ when boundary conditions are not considered.
The same is then true for the solution of the discretized problem $\bm u$ and hence it follows that $\bm A$ is singular.
This is verified numerically by finding that $\textrm{rank}(\bm A) < n, \ \bm A \in \mathbb{R}^{n \times n}$.
\end{remark}

\subsection{Boundary conditions}
As stated in \eqref{poission-problem} we have homogenous Dirichlet conditions on parts of the domain; $\restr{u}{\partial \Omega_D} =  0 $. This has not been considered in the given construction of $\bm A,\bm f$, which could be denoted as a proto problem. Hence, to find the solution $u \in X_h$ we will use the so called Big Number approach \cite[p. 16]{Lecture_Note_4} to enforce the Dirichlet boundary conditions. With $\tilde n$ inner nodes, for nodes on the boundary $(x_i,y_i), i= \tilde n +1,\cdots, n$ we set:


\begin{equation*}
    \begin{aligned}
    A_{ii} &= \frac{1}{\epsilon}, \, \epsilon << 1 \\
    f_i &= 0
    \end{aligned}
\end{equation*}
which for a sufficiently small $\epsilon$ enforces the boundary conditions. In this way Dirichlet boundary conditions are enforced essentialy by restricting the function space. 

Neumann boundary conditions are enforced as in Equation \eqref{neumann-conditions} in the nodes $i$ where $(x_i,y_i) \in \partial \Omega_N$ yielding the discretized $l$ one the form
\begin{align}
    f_i = l(\phi_i) + g(x_i,y_i)l(\phi_i).
\end{align}


\section{The elasticity problem } \label{seq-elasticity}

\subsection{Boundary value problem}
Let $\boldsymbol{u} : \Omega \rightarrow \mathbb{R}^2$ be the displacement of each point $\boldsymbol{x} \in \Omega$. Then the strain tensor is defined as 
\begin{equation} \label{eq-strain-tensor}
    \boldsymbol{\epsilon}(\boldsymbol{u}) = \frac{1}{2}(\nabla \boldsymbol{u} + \nabla \boldsymbol{u}^\intercal)
\end{equation}
Also the Generalized Hooks Law states that
\begin{equation}
    C \boldsymbol{\overline{\epsilon}}  = \boldsymbol{\overline{\sigma}};
    \label{hooks-law}
\end{equation}
where 
\begin{equation*}
    \boldsymbol{\overline{\epsilon}} =\begin{aligned}[t]
    \begin{bmatrix}
        \epsilon_{1 1} \\
        \epsilon_{2 2} \\
        \epsilon_{1 2} + \epsilon_{2 1}
    \end{bmatrix}
    \end{aligned}
    \qquad \text{and} \qquad
    \begin{aligned}[t]
        \boldsymbol{\overline{\sigma}} = \begin{bmatrix}
            \sigma_{1 1} \\
            \sigma_{2 2} \\
            \sigma_{1 2} 
        \end{bmatrix}
    \end{aligned},
\end{equation*}
and C is a linear transformation.
Finally, for a static body the Cauchy momentum equation results in 
\begin{equation} \label{eq-cauchy-momentum}
    \nabla \cdot \boldsymbol{\sigma} = -\boldsymbol{f},
\end{equation}
where $\boldsymbol{f}$ is the force per area. Applying boundary conditions 
\begin{align}
    \boldsymbol{u}  = \boldsymbol{g} \quad \text{on} \quad \Omega_D   \\
    \boldsymbol{\sigma} \cdot \boldsymbol{\hat{n}}  =  \boldsymbol{h} \quad \text{on} \quad \Omega_N,
    \label{elasticity-bc}
\end{align}
result in the corresponding boundary value problem. 
\subsection{Weak formulation}
\begin{proposition}
    % Skal vi ha med lifing functions her ??. Ikke enda?
    The weak formulation of \eqref{eq-cauchy-momentum} with homogeneous dirichlet conditions is given by:  
    \begin{equation*}
        \begin{aligned}
        &\text{find } u \in X : 
        &a(u,v) = l(v) \ \forall v \in H_0^{1}(\Omega), 
        \label{weak_formulation_elastic}
        \end{aligned}
    \end{equation*}
    where $a$ is the bilinear functional and $l$ the linear functional defined as 
    \begin{equation*}
        \begin{aligned}
            a(u,v) &= \int_{\Omega} \boldsymbol{\overline \epsilon}(\boldsymbol v)^\intercal C\boldsymbol{\overline \epsilon}(\boldsymbol u)  \, dA
            \\
            l(v) &= \int_\Omega{\boldsymbol{v}^\intercal \boldsymbol{f}} \,dA  + \int_{\partial \Omega_N}{\boldsymbol{v}^\intercal \boldsymbol h} \,dS,
        \end{aligned}
    \end{equation*}
    and $ X=   H^1_{\partial \Omega_D } \times H^1_{\partial \Omega_D } $.
\end{proposition}
\begin{proof}
Taking the dot product of  Equation \eqref{eq-cauchy-momentum} with a test function $\boldsymbol{v} = [v_1, v_2]^\intercal$, then integrating over $\Omega$, the left side becomes
\begin{equation} \label{eq-cauchy-expand}
    \int_\Omega{\boldsymbol{v}^\intercal \nabla \cdot \boldsymbol{\sigma}} \,dA = \sum_{i = 1,2}\left(\int_\Omega{\boldsymbol{v_i} \nabla \cdot \boldsymbol{\sigma_i}} \,dA \right),
\end{equation}
where $\boldsymbol{\sigma_i}$ correspons to the i-th row of the stress tensor $\boldsymbol{\sigma}$. 
Then applying the Divergence Theorem and $\nabla \cdot (v_i \boldsymbol{\sigma_i})  = v_i \nabla \cdot \boldsymbol{\sigma_i} + \boldsymbol{\sigma_i} \cdot \nabla v_i$ results in 
\begin{equation} \label{eq-div-theorem-to-Cauchy-momentum}
    \int_\Omega{{v_i} \nabla \cdot \boldsymbol{\sigma_i}}  \,dA = -\int_\Omega{\boldsymbol{\sigma_i}^\intercal \nabla v_i}  \,dA +  \int_{\partial \Omega}{v_i \boldsymbol{\sigma_i} \cdot \boldsymbol{\hat{n}} }  \,dS.
\end{equation}
Interperating $\boldsymbol{v}$ as displacement we get the corresponding strain tensor $\boldsymbol{\epsilon} (\boldsymbol v)$. Expanding each term of Equation\eqref{eq-cauchy-expand} like Equation \eqref{eq-div-theorem-to-Cauchy-momentum} and inserting the strain tensor of $\boldsymbol{v}$ we get
\begin{equation}\label{eq-elastic-weak-left}
    \int_\Omega{\boldsymbol{v}^\intercal \nabla \cdot \boldsymbol{\sigma}} \,dA = -\int_\Omega{\tr[ \boldsymbol{\epsilon}(\boldsymbol v)\boldsymbol{\sigma}(\boldsymbol u)]} \,dA + \int_{\partial \Omega}{\boldsymbol{v}^\intercal \boldsymbol{\sigma}}\boldsymbol{\hat{n} } \,dS.
\end{equation}
Thus Equation \eqref{eq-cauchy-momentum} becomes
\begin{equation} \label{eq-almost-elastic-weak}
    \int_\Omega{\tr[ \boldsymbol{\epsilon}(\boldsymbol v)\boldsymbol{\sigma}(\boldsymbol u)]} \,dA  = \int_\Omega{\boldsymbol{v}^\intercal    \boldsymbol{f}} \,dA + \int_{\partial \Omega}{\boldsymbol{v}^\intercal \boldsymbol{\sigma}}\boldsymbol{\hat{n} } \,dS.
\end{equation}
By expanding all the matix operations Equation \eqref{eq-almost-elastic-weak} can be written as 
\begin{equation}
    \sum_{i = 1,2,3} \left[\int_\Omega{ \overline{\sigma}_i(\boldsymbol v) \overline{\epsilon}_i(\boldsymbol v)}   \,dA \right] = \sum_{i = 1}^2 \left[\int_\Omega{ v_i f_i}   \,dA \right] + \sum_{i = 1}^{2}{{\sum_{j = 1}^{2} \left[\int_{\partial \Omega}{ v_i \sigma_{ij} \hat{n}_j}   \,dS \right]}}. 
\end{equation}
The left side of this equation is the integral over $\boldsymbol{\overline \epsilon}(\boldsymbol v)^\intercal \boldsymbol{\overline \sigma}(\boldsymbol u) = \boldsymbol{\overline \epsilon}(\boldsymbol v)^\intercal C\boldsymbol{\overline \epsilon}(\boldsymbol u)$. Therefore Equation \eqref{eq-almost-elastic-weak} is equivalent to the more compact form
\begin{align}
    \int_\Omega{ \boldsymbol{\overline \epsilon}(\boldsymbol v)^\intercal C\boldsymbol{\overline \epsilon}(\boldsymbol u)} \,dA  &=\int_\Omega{\boldsymbol{v}^\intercal \boldsymbol{f}} \,dA + \int_{\partial \Omega}{\boldsymbol{v}^\intercal \boldsymbol{\sigma}}\boldsymbol{\hat{n} } \,dS  \\
    &= \int_\Omega{\boldsymbol{v}^\intercal \boldsymbol{f}} \,dA + \underbrace{\int_{\partial \Omega_D}{\boldsymbol{v}^\intercal \boldsymbol{\sigma}}\boldsymbol{\hat{n} } \,dS}_{= \ 0} + \int_{\partial \Omega_N}{\boldsymbol{v}^\intercal \boldsymbol h} \,dS,
\end{align}
where the Neumann boundary conditions where incooroprated into the last step.

Since $\bm v$ was arbitrary the weak formulation follows. Furthermore, all $\bm u$ and $\bm v$ must belong to a space such that $\boldsymbol{\overline \epsilon}(\boldsymbol v)^\intercal C\boldsymbol{\overline \epsilon}(\boldsymbol u) \in {L^1}(\Omega)$. By linearity of the integral then $\nabla \bm u$ and $\nabla \bm v$ must be in $\left[{L^2}(\Omega)\right]^{2 \times 2}$; thus $\bm u$ and $\bm v$ must be in $ H^1_{\partial \Omega_D } \times H^1_{\partial \Omega_D } $
\end{proof}
\begin{remark}
    The case for nonhomogenous Dirichlet boundary conditions can be solved by lifting functions.
\end{remark}
\subsection{Galerkin projection}
Again, lets find finite dimesional solutions $u_h \in X_h \subset X$
 %TODO: Comment that \varphi is piecewise linear
\begin{proposition}
    The finite dimensional equivalent of the weak formulation given in Proposition 4 is given by


        \begin{align}
        &\text{find } \bm u_h \in \mathbb{R}^n : \\
        & A \bm u_h = \bm b_h,
        \end{align}


    where 


        \begin{align}
        A = [A_{ij}] &= \int_\Omega \overline\epsilon (\bm \varphi_i)^T C \overline \epsilon (\bm \varphi_j) \, dA \\
        \bm b_h = [b_j] &= \int_\Omega \bm \varphi_j^T \bm f \, dA + \int_{\partial \Omega_N} \bm \varphi_j^T \bm h \, d\bm S,
        \end{align}


    where each node $\hat i$ has two test functions
    
 
        \begin{align}
        \bm \varphi_{\hat i,1} (\bm x) = 
         \begin{bmatrix}
            \varphi_{\hat i}  \\
            0 
        \end{bmatrix},
        &
        \bm \varphi_{\hat i,2} (\bm x) = 
        \begin{bmatrix}
            0  \\
            \varphi_{\hat i} 
        \end{bmatrix} 
    \end{align}

    where for $\bm \varphi_i$ we have a single index $i = 2\hat i + d$ where $i$ is the node number and $d$ is the vector component of the function.
\end{proposition}

\begin{proof}
    As $\bm u_h,\bm v_h \in X_h$ are projections into a finite subspace, we can write the functions as an expansion of basis functions $\bm \varphi$. Hence we can write 
    $\bm v_h = \bm \varphi_j$ and $\bm{ \hat u_h} = \sum_i u_i \bm \varphi_i, u_i \in \mathbb{R}$. As $\bar \epsilon( \cdot )$ is linear we get equations for $j = 1,\cdots,n$ on the form

    \begin{align*}
        a(\bm{ \hat u_h},\bm v_h) &= \int_\Omega{ \boldsymbol{\overline \epsilon}(\bm v_h)^\intercal C\boldsymbol{\overline \epsilon}(\bm{ \hat u_h})} \,dA  \\
        &=\sum_i u_i  \int_\Omega{ \boldsymbol{\overline \epsilon}(\boldsymbol \varphi_j)^\intercal C  \boldsymbol{\overline \epsilon}(\boldsymbol \varphi_i)} \,dA \\
        &= \sum_i u_i a(\bm \varphi_j,\bm \varphi_i) \\
        &= A_j \bm u_h \\
    \end{align*}

    where $\bm u_h = [u_1,u_2,\cdots,u_n]^\intercal$ and $A_j = [a(\bm \varphi_j,\bm \varphi_1),\cdots,a(\bm \varphi_j,\bm \varphi_n)]$ finally we have equations for $j = 1,\cdots,n$ for the linear form

    \begin{align*}
            l(\bm v_h)&= \int_\Omega{\bm v_h^\intercal  \boldsymbol{f}} \,dA + \int_{\partial \Omega_N}{\bm v_h^\intercal \boldsymbol h} \,dS \\
            &= \int_\Omega{\boldsymbol{\varphi_j}^\intercal  \boldsymbol{f}} \,dA + \int_{\partial \Omega_N}{\boldsymbol{\varphi_j}^\intercal \boldsymbol h} \,dS \\
            &= b_j\\
    \end{align*}
    
    Taking $A_j$ as rows in a matrix $A$, $\bm b_h = [b_1,\cdots,b_n]^\intercal$ and by the weak formulation \eqref{weak_formulation_elastic} we get

    \begin{align*}
        a(\bm u_h,\bm v_h) &= l(\bm v_h), \\
        A_j \bm u_h &= b_j, \,\, j = 1,\cdots,n\\
        \implies A\bm u_h &= \bm b_h.
    \end{align*}

\end{proof}


\subsection{Analytical solution}


    We consider an analytical solution $u$ to the elasticity problem \eqref{eq-cauchy-momentum} with homogeneous Dirichlet boundary conditions.

\begin{proposition} \label{prop-eleastic-test-case}
    $u: \mathbb{R}^2 \rightarrow \mathbb{R}^2$ such that

    \begin{equation} \bm u =
        \begin{bmatrix}
            (x^2-1)(y^2-1), \\
            (x^2-1)(y^2-1)
        \end{bmatrix},
    \end{equation}

    is a solution to the problem

    \begin{align*}
        \nabla \bm \sigma (\bm u) &= -\bm f \ \text{in} \ \Omega \\
        \bm u &= \bm 0 \ \text{on}  \ \partial  \Omega,
    \end{align*}

    where

    \begin{align*}
        \bm f = \frac{E}{1-\nu^2}   
        \begin{bmatrix}
            -2y^2 + x^2(\nu - 1) -2xy(\nu +1) + 3 - \nu, \\
            -2x^2 + y^2(\nu - 1) -2xy(\nu +1) + 3 - \nu 
        \end{bmatrix}^T,
    \end{align*}

    and $\Omega = \{ (x,y) : \max (|x|,|y|) \leq 1 \}$ is the reference square $(-1,1) \times (-1,1)$.

\end{proposition}

\begin{proof}
By Hooks law \eqref{hooks-law} have that $C \bar{\epsilon} = \bar{\sigma}$ where 

\begin{equation*}
    C = \underbrace{\frac{E}{1-\nu^2}}_{=c}\begin{bmatrix}
        1&\nu&0\\
        \nu&1&0\\
        0&0&\frac{1-\nu}{2}
    \end{bmatrix},
\end{equation*}

\noindent where Youngs modulus $E$ and the Poisson ration $\nu$ is material specific constants describing stiffness and the ratio of compression vs expansion of the material being modelled. As the components of $\bm u$ are identical we write $\bm u = [u,u]^T$ where we denote derivatives by $\frac{\partial u}{\partial x} = u_x, \, \frac{\partial^2 u}{\partial x \partial y} = u_{xy} \cdots$,
we get

\begin{align}
    \nabla\bm\sigma(u) &= 
    c\begin{bmatrix}
        u_{xx} + u_{yy}\frac{1-\nu}{2} + u_{xy}(\nu + \frac{1-\nu}{2}),\\
        u_{yy} + u_{xx}\frac{1-\nu}{2} + u_{xy}(\nu + \frac{1-\nu}{2})
    \end{bmatrix}^T\\
    &= c\begin{bmatrix}
        2y^2 - x^2(\nu - 1) +2xy(\nu +1) - 3 + \nu, \\
        2x^2 - y^2(\nu - 1) +2xy(\nu +1) - 3 + \nu
    \end{bmatrix}^T\\
    &= -\bm f^T.
\end{align}

On the boundary $\partial \Omega = (-1,1)\times(-1,1)$ we get $\bm u(x,y) = \bm 0 \, \forall (x,y) \in \partial \Omega$.
Hence we find that $\bm u$ fullfills all requirements in proposition 6.  
\end{proof}
    


\section{Numerical experiments}
\subsection{Mesh}

For the poission problem, we solve the problem on a regular disc mesh, which is generated by giving the number of nodes in the domain $N_{tot}$, yielding  $(x_i,y_i) \in \Omega, i = 1,\cdots,N_{tot}$. Meshes for $N_{tot} = 16,64,256$ could be seen in Figure \ref{plot-meshes}.

For the elasicity problem we use a plate mesh generated by giving the number of nodes in one spatial dimension $N_{1D}$. Examples are found in Figure \ref{plot-meshes-plate}.




\begin{figure}[ht]
    \begin{subfigure}[t]{1\textwidth}
        \centering
        \includegraphics[width=\textwidth]{figures/plot_meshes.pdf}
    \caption{The generated mesh for a disc with 16, 64 and 256 nodes. }
    \label{plot-meshes}
    \end{subfigure}\qquad
    \begin{subfigure}[b]{1\textwidth}
        \centering
        \includegraphics[width=\textwidth]{figures/plot_meshes_plate.pdf}
    \caption{The generated mesh for a square plate with 4, 8 and 16 nodes in one spatial direction.}
    \label{plot-meshes-plate}
    \end{subfigure}
    \label{meshes}
    \caption{Meshes for solving the Poission problem and the elasicity equation.}
\end{figure}





\begin{figure}[ht]
    \begin{subfigure}[b]{0.5\textwidth}
        \centering
            \includegraphics[width=\linewidth]{figures/plot_homogeneous_dirichlet.pdf}
            \caption{Solution and error with Dirichlet boundary conditions}
        \label{solution-error-dirichlet}
    \end{subfigure}\qquad
    \begin{subfigure}[b]{0.5\textwidth}
        \centering
            \includegraphics[width=\linewidth]{figures/plot_mixed_neumann.pdf}
            \caption{Solution and error with mixed boundary conditions}
        \label{solution-error-mixed}
    \end{subfigure}
    \label{plot-solution-error}
    \caption{The poission problem solved on a disc mesh with $N = 1024$ nodes.
    The top plot displays the numerical solution from the Galerkin projection,
    while the bottom figure displays the error in the nodal points.}
\end{figure}


\subsection{Numerical integration}
To evaluate the integrals in the weak formulation gaussian quadrature is used. For a spessific number of nodes $n+1$, we find the polynomal of degree $n+1$ orthogonal to all polynomals of lesser degree. The integrand is then interpolated using the zeroes of this orthogonal polynomal as nodes. For each node $z_q$ we already know the integral of the assosiated lagarange basis. Thus by the linearety of the integral we have predetermined wheights $\rho_q$  for each node $z_q$. Thus on a domain $\hat{\Omega}$ we get the quadrature rule
\begin{equation}
    \int_{\hat{\Omega}} g \, d\hat{\Omega} = \sum_{q=0}^{n}{\rho_q g(z_q)}.
\end{equation}
For a domain $\Omega$ we can then scale the integral. Finally it can be shown that this quadrature rulse has degree of exacteness $2n+1$.


\subsection{Poisson with Homogeneous Dirichlet boundary conditions}
The finite element method described is tested on the Poisson problem with homogeneous Dirichlet conditions and $f$ given by Equation \eqref{f-analytic}. 
By Proposition \ref{prop-analytical-solution} the solution is known and is used as a reference. 
Numerical experiments with increasing degrees of freedom results in lower relative maximal error, as seen in Figure \ref{convergence-dirichlet}. The numerical solution for $N = 1024$ nodes is shown in Figure \ref{solution-error-dirichlet}
together with a plot of how the error $e_{i} = U_{i} - u(x_i,y_i)$ is located on our domain $\Omega$. 
The relative maximal error is given by

\begin{equation*}
    e_{rel,max} = \max_{1\leq i \leq n} \frac{|e_i|}{\| u(x,y)\|_{\infty}} = \max_{1\leq i \leq n} |e_i|.
\end{equation*}

\subsection{Poisson with mixed boundary conditions}
The finite element method is used to solve the Poisson problem given by \eqref{poission-problem} with $f$ given by Equation \eqref{f-analytic}, $g$ given by Equation \eqref{g-analytic} and $\partial \Omega_N = \{ x,y \in \partial \Omega : y> 0\}$. By Proposition \ref{analytical-solution-neumann} the solution is the is the same as for Homogeneous Dirichlet boundary conditions. Numerical experiments with increasing degrees of freedom results in lower relative maximal error, as seen in Figure \ref{convergence-neumann}. The numerical solution for $N = 1024$ nodes is plotted in Figure \ref{solution-error-mixed} together with a plot of how the error $e_{i,j} = U_{i,j} - u(x_i,y_j)$ is located on our domain $\Omega$. 

\begin{figure}[t]
\begin{subfigure}[t]{0.5\linewidth}
        \centering
            \includegraphics[width=\linewidth]{figures/convergence_homogeneous_dirichlet.pdf}
        \caption{Error with Dirichlet boundary conditions}
        \label{convergence-dirichlet}
    \end{subfigure}\qquad
    \begin{subfigure}[t]{0.5\linewidth}
        \centering
            \includegraphics[width=\linewidth]{figures/convergence_mixed_neumann.pdf}
        \caption{Error with mixed boundary conditions}
        \label{convergence-neumann}
    \end{subfigure}
    \label{fig-convergence}
    \caption{Relative error of numerical solutios to the poission problem for multiple mesh discs. The number of nodes given by $N_n = \{ 2^n\}_{n=4}^{10}$
    where $N_{min} = 16$ and $N_{max} = 1024$.}
\end{figure}


\subsection{Elasticity with Homogeneous Dirichlet boundary conditions}
We implement the finite elemenet method for the elasticity problem described in Section \ref{seq-elasticity}. The method is applied to the boundary value problem described in Proposition \eqref{prop-eleastic-test-case}, and the first displacement component is shown in Figure \ref{solution-error-dirichlet-elastic-x}. Youngs modulus $E$ and the Poisson ration $\nu$ could be specified as to represent a given material. In this case we set $E = 200$ and $\nu = 0.30$ which describe properties of steel. 

The recovered stress for each node is then calculated by averageing the stress over all neighbouring elements. For the previous numerical solution the recovered stress is shown in Figure \ref{solution-error-dirichlet-elastic-sigma-xx}. 



\begin{figure}[ht]   
    \begin{subfigure}[b]{0.5\textwidth}
        \centering
            \includegraphics[width=\linewidth]{figures/plot_homogeneous_dirichlet_elastic_x.pdf}
            \caption{One component of the displacement, $u_x$ and error of the elasticity problem with Dirichlet boundary conditions for $N_{1D} = 16$.}
        \label{solution-error-dirichlet-elastic-x}
    \end{subfigure}\qquad
    \begin{subfigure}[b]{0.5\textwidth}
        \centering
            \includegraphics[width=\linewidth]{figures/plot_homogeneous_dirichlet_elastic_sigma_dim_0_0.pdf}
            \caption{One component of the recovered stress, $\sigma_{xx}$, and the absolute error comared to the analytical solution, for $N_{1D} = 16$.}
        \label{solution-error-dirichlet-elastic-sigma-xx}
    \end{subfigure}

    \caption{The elasticity problem problem solved on the unit square mesh with $N = 256$ nodes. The top figures displays displacement $u_x$ to the left and the recovered stress $\sigma_{xx}$ to the right. The bottom figures displays the error in the nodal points for displacement and stress. For the displacement, we have the error $e_{x,i,j} = U_{x,i,j} - u_x(x_i,y_j)$.}
\end{figure}

\subsection*{Measure of performance}

Assuming the analytical solution to \eqref{weak_formulation_elastic} is unknown, a way to estimate the convergence rate is to solve \eqref{weak_formulation_matrix} using very small elements with size $h_{fine}$, finding the corresponding numerical solution $\bm U_f$, and estimate the error by finding solutions with larger element sizes $h_i >> h_{fine}$, yielding solutions $\bm U_{h,i}$. Hence we find the estimated error in max norm $\hat e_{i,r} = \|U_{f,r} - U_{h,i,r} \|_{\infty}$, where $r\in \{x,y\}$. Let $h_i \in \{\frac{1}{2^i}\}_{i=1}^4$, and $h_{fine} = \frac{1}{2^6}$ yielding $\{\hat e_{i,r}\}_{i=1}^4$. The max norm is used, because, if the $l_2$ norm was used, one would experience a slight increase in the error with increasing number of nodes. Finding the slope of the linear regression curve of the points $\{\log h_i,\log  \hat e_{i,r}\}_{i=1}^4$, we obtain the rate of convergence for our numerical scheme. 

By using the Time functionality of Python, one can find the runtime, the number of seconds it takes to run the elasticity solver, $t_i$ given element size $h_i$. A plot of $\{\log h_i,\log \hat e_{i,x}\}_{i=1}^4$ is found in Figure \ref{error-convergence-elastic}, and we find the numerical convergence rate given the max norm to be $2.46$. Furthemore the runtime over the different element sizes is found in Figure \ref{time-convergence-elastic} and the slope of the linear regression curve for the runtime is $-2.49$, meaning we get increasing runtime with decreasing element size, which is as expected. It should also be noted that since we solve the equation for a two dimensional vectorfield on a two dimensional grid, reducing the element size (in one dimension), increases the number of nodes quadratically, as $h_i = \frac{1}{2N_{1D}} = \frac{1}{4N_{2D}^2}$.

\begin{figure}[t]
    \begin{subfigure}[t]{0.5\linewidth}
            \centering
                \includegraphics[width=\linewidth]{figures/decreasing_h_error.pdf}
            \caption{Error approximation $\hat e_{i,x} = \|U_{f,x} - U_{h,i,x}\|_{\infty}$ between fine mesh with size $h_f$ and coarser meshes with sizes $h_i$.}
            \label{error-convergence-elastic}
        \end{subfigure}\qquad
        \begin{subfigure}[t]{0.5\linewidth}
            \centering
                \includegraphics[width=\linewidth]{figures/decreasing_h_runtime.pdf}
            \caption{Runtime, number of seconds to solve the elasticity problem on meshes with element sizes $h_i$ including the fine mesh with $h_{fine}.$}
            \label{time-convergence-elastic}
        \end{subfigure}
        \label{fig-convergence-elastic}
        \caption{The elasicity problem is first solved on a fine mesh with element size $h_{fine} = \frac{1}{2^6}$. Furtheremore we solve the problem on coarser meshes with $h_i = \frac{1}{2^i}, i = 1,2,3,4$, yielding an approximate method to check for convergence.}
    \end{figure}


\subsection*{Optimize runtime of solver}

Below follows a discussion on different ways to optimize the implemented finite element solver aiming to reduce the runtime. First it should be noted that the matrix $A_h$ in Equation \ref{weak_formulation_matrix} is sparse and by using a sparse matrix storage format, including a sparse solver for linear systems, we can obtain a quite large reduction in computational time.

By using a profiler to find the runtime of the different functions in the code, two remarks should be made. First, when a sparse solver is used for the linear system $A \bm u_h = \bm b_h$, the most time consuming task is the assembly of $A,\bm b_h$. Secondly, within the assembly of $A,\bm b_h$, approximating the integrals over the basis functions is done with gaussian quadrature. In both problems, we use linear basis functions and given the weak formulations, we obtain integrals over constants. In the elasticity problem we furthermore solve on a grid in which the elements have same shape rotated in different directions. Thus we could find the integral for each direction; drasticly reducing the number of operations required to construct the $A$ matrix. This would yield a scheme that is less flexible to use with basis functions of higher order, however one would enable an increase in the speed of the computations. 
   
\section{Conclusion}

Using the finite difference method for finding approximate solutions to the Poisson equation, solutions compares well with the analytic solution from Equation \eqref{analytical-solution}. For 1024 nodes, the maximal relative error is approximately $e_{rel,max} \approx 0.02$ for homogeneous Dirichlet boundary conditions, and approximately $e_{rel,max} \approx 0.03$ for mixed boundary conditions. In addition, the error decreases with the number of nodes increases. This indicates that the method is convergent.

For the elasticity problem, the error was approximated by comparing solutions on a fine mesh with coarser meshes. In this case a numerical convergence rate of $2.46$ was observed, when decreasing the element size, indicating that the method is convergent. However, decreasing error comes with the cost of increased computational time, where the slope of the logarithmic plot (Figure \ref{time-convergence-elastic}) of time usage for different stepsizes was found to be approximately $-2.49$. This indicates that for this scheme, used to solve the elasticity problem, error decreases at almost the same rate as computational time increases, when decreasing the element size.
	

\bibliography{ref}

\end{document}


